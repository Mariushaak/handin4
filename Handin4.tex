\documentclass[11pt]{amsart}
\usepackage{amsfonts}
\usepackage{amssymb}
\usepackage{graphicx}
\graphicspath{ {\string~/Documents/LaTeX/} }
\usepackage{url}
\usepackage{amsmath}
\usepackage{fancyvrb}
\usepackage[margin=1in]{geometry}

\title{Hand in Module 4}
\author{Ole K Larsen, Marius Haakonsen}

\begin{document}

\maketitle

\section{Task 1}

The demo of interpolation will show a circle shape moving from the top left corner to the bottom right corner of the screen.
The shape will begin as a small, dark line, and end up being a large, lightly colored oval shape. \\

Using the 'float lerp' function we can return the interpolated version of P0 and P1 based on the variable 'tid'. \\

\begin{verbatim}

  float lerp(float tid, float P0, float P1) { // Returns interpolated float.
      return (P1 - P0) * tid + P0;
  }

\end{verbatim}

The following variables are used to interpolate to and from. \\

\begin{verbatim}

float minX = 5.0f;
float minY = 5.0f; // X and Y pos. for where the circle begins

float maxX = WIDTH - 150.0f;
float maxY = HEIGTH - 150.0f; // X and Y pos. for where the circle ends

float minRadius = 5.0f;     // Beginning radius
float maxRadius = 300.0f;   // Ending radius

float minCol = 1.0f;    // Beginning color
float maxCol = 255.0f;  // Ending color

float minScl = 0.0f;  // Beginning scale
float maxScl = 1.0f;  // Ending scale

\end{verbatim}

Changing the poisition utilizes CircleShape.setPosition, taking an x and y value. \\

\begin{verbatim}

  // Interpolating position
  c.setPosition(lerp(t, minX, maxX),lerp(t, minY, maxY));

\end{verbatim}


\begin{center}
\includegraphics[scale=0.25]{c1.png}
\end{center}

Changing the size utilizes CircleShape.setRadius, taking a single value that slowly increases. \\

\begin{verbatim}

  // Interpolating size
  c.setRadius(lerp(t, minRadius, maxRadius*0.75));

\end{verbatim}

\begin{center}
\includegraphics[scale=0.25]{c2.png}
\end{center}

Changing the color utilizes CircleShape.setFillColor. This function takes 3 variables.
In this example they are scaled with +50 and +100 to vary the color from only white to black. \\

\begin{verbatim}

  // Interpolating color
  c.setFillColor(sf::Color(lerp(t, minCol+50, maxCol),
                           lerp(t, maxCol, minCol),
                           lerp(t, minCol+100, maxCol)));

\end{verbatim}

\begin{center}
\includegraphics[scale=0.25]{c3.png}
\end{center}

Changing the shape utilizes CircleShape.setScale, taking 2 values: the size in the x direction and the size in the y direction. \\

\begin{verbatim}

  // Interpolating shape
  c.setScale(lerp(t, minScl, maxScl)*5, lerp(t, minScl, maxScl)+minX/4);

\end{verbatim}

\begin{center}
\includegraphics[scale=0.25]{c4.png}
\end{center}

The time variable is incremented by 0.001 each game loop, and reseting when t exceeds 1. \\
\begin{verbatim}

  t += 0.001f;
    if(t > 1) {
        t = 0;
    }

\end{verbatim}

\section{Task 4}

\textbf{a)}\\\\

Defining an array of lines 'lines', capable of storing a maximum 1025 lines. \\
The first and last line is set to the middle of the screen. \\\\

\begin{verbatim}

  sf::VertexArray lines(sf::LinesStrip, WIDTH+1);

      lines[0].position = sf::Vector2f(0,HEIGTH/2);
      lines[WIDTH].position = sf::Vector2f(WIDTH, HEIGTH/2);

\end{verbatim}

A float array of datapoints is declared to store 1025 values. \\

The first and last value in the data structure is set to 0.0f.\\\\

\begin{verbatim}

      float data[WIDTH+1];
      data[0] = 0.0f;
      data[WIDTH] = 0.0f;

\end{verbatim}

The first loop happens 512 times, dividing stepsize (=1024) by 2 each time.
The nested loop with 'int i' moves 1 stepsize each increment. \\\\

\begin{verbatim}

  for (int stepsize = WIDTH; stepsize > 1; stepsize/=2) {
      for (int i = stepsize/2; i < WIDTH; i += stepsize) {

\end{verbatim}

The following bits of code happens inside the nested for-loops. \\

Creating a random value 'val' and scaling it to a suitable size based on screen heigth. \\\\

\begin{verbatim}

      val = dis(gen);
      val *= HEIGTH/2.5;

\end{verbatim}

The data[i] value will function as the corresponding y-value for line[i], using
the midpoint displacement algorithm. \\

When exiting the nested loop, alpha (0.5) is divided by beta (2.0). \\\\

\begin{verbatim}

              data[i] = ((data[i-stepsize/2] + data[i+stepsize/2])/2 + alpha*val);

              lines[i].position = sf::Vector2f(WIDTH/(WIDTH/2)*i, data[i]+HEIGTH/2);

          } alpha /= beta;
      }

\end{verbatim}

\end{document}
