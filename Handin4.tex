\documentclass[11pt]{amsart}
\usepackage{amsfonts}
\usepackage{amssymb}
\usepackage{graphicx}
\usepackage{url}
\usepackage{amsmath}
\usepackage{fancyvrb}
\usepackage[margin=1in]{geometry}

\title{Hand in Module 4}
\author{Ole K Larsen, Marius Haakonsen}

\begin{document}

	\maketitle

	\section{Task 1}
	\section{Exercise 1:}

	\textbf{a)}\\


	\section{Exercise 2:}




\section{Task 2}

\textbf{a)}\\\\

\textbf{The program: }

\begin{verbatim}

import numpy as np

m = np.array([[1,2,3,4],[3,4,5,6],[3,2,1,1]])

U, d, V = np.linalg.svd(m, full_matrices = False)
D = np.diag(d)

  # by printing the matrices multiplied by each other we can see
  # that the product of them is 'M'

print(U @ D @ V)

\end{verbatim}



\textbf{The output:}

\begin{verbatim}


[[ 1.  2.  3.  4.]
[ 3.  4.  5.  6.]
[ 3.  2.  1.  1.]]

\end{verbatim}

The program first runs a singular value decomposition on the matrix M. \\
By multiplying all the matrices we got from the decomposition, we get the initial matrix M, as we can see from the output of the program. \\\\


\textbf{b}\\\\

Tested the code by implementing different values. \\\\


\textbf{c)}\\\\

The following sources was used for instructions.\\\\

\bibitem{Wiki}
Wikipedia, k-NN
\\\texttt{https://en.wikipedia.org/wiki/K-nearest_neighbors_algorithm}
\end{thebibliography}
\bibitem{Tut}
Tutorial to implement k-NN
\\\texttt{https://machinelearningmastery.com/tutorial-to-implement-k-nearest-neighbors-in-python-from-scratch/}
\end{thebibliography}
\\\\

Pseudocode:\\\\

Using a getNeighbours function to count the variable k nearest, at the distance d.\\\\

\begin{verbatim}

def getNeighbours(k, d):
  set distance to equal d.
  set counting variable to equal k.
	  define a neighbour x.
    for x in range d, and less or equal to k {
      neighbours.append(x)
    }

  return neighbours.

\end{verbatim}

\end{document}
